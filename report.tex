\documentclass[12pt]{report}   % list options between brackets
\usepackage{cwpuzzle}              % list packages between braces
\usepackage{fullpage}
\usepackage{hyperref}
\usepackage{listings}
\usepackage{sidecap}

\lstset{ %
        language=Haskell,
        basicstyle=\footnotesize,
        frame=single}

% type user-defined commands here

\begin{document}

\title{Haskword: Generating Word Searches}   % type title between braces
\author{Thomas Schreiber}         % type author(s) between braces
\date{March 20, 2012}    % type date between braces

\begin{minipage}[h]{\textwidth}
  \maketitle

  \vspace{12pt}

  \begin{Puzzle}{14}{14}%
        |[][S]R |[][S]A |[][S]S |[][S]H |[][S]E |[][S]L |[][S]P |[][S]S |[][S]L |[][S]M |[][S]N |[][S]C |[][S]D |[][S]L |.
        |[][S]N |[][S]C |[][S]R |[][S]H |[][S]A |[][S]S |[][S]K |[][S]E |[][S]L |[][S]L |[][S]R |[][S]T |[][S]S |[][S]A |.
        |[][S]H |[][S]H |[][S]L |[][S]U |[][S]N |[][S]N |[][S]R |[][S]C |[][S]A |[][S]L |[][S]U |[][S]N |[][S]L |[][S]A |.
        |[][S]U |[][S]E |[][S]M |[][S]U |[][S]R |[][S]L |[][S]R |[][S]P |[][S]F |[][S]R |[][S]E |[][S]I |[][S]T |[][S]U |.
        |[][S]C |[][S]D |[][S]M |[][S]N |[][S]A |[][S]O |[][S]L |[][S]S |[][S]I |[][S]E |[][S]S |[][S]N |[][S]A |[][S]A |.
        |[][S]E |[][S]C |[][S]Y |[][S]N |[][S]T |[][S]C |[][S]U |[][S]R |[][S]I |[][S]P |[][S]A |[][S]M |[][S]U |[][S]T |.
        |[][S]L |[][S]R |[][S]R |[][S]C |[][S]C |[][S]I |[][S]U |[][S]P |[][S]G |[][S]N |[][S]A |[][S]L |[][S]R |[][S]E |.
        |[][S]D |[][S]A |[][S]N |[][S]O |[][S]M |[][S]T |[][S]A |[][S]R |[][S]F |[][S]S |[][S]L |[][S]R |[][S]A |[][S]L |.
        |[][S]N |[][S]U |[][S]E |[][S]R |[][S]E |[][S]F |[][S]N |[][S]U |[][S]R |[][S]C |[][S]I |[][S]M |[][S]K |[][S]K |.
        |[][S]F |[][S]T |[][S]H |[][S]A |[][S]T |[][S]S |[][S]E |[][S]I |[][S]F |[][S]Y |[][S]R |[][S]A |[][S]S |[][S]L |.
        |[][S]L |[][S]L |[][S]A |[][S]N |[][S]O |[][S]I |[][S]T |[][S]C |[][S]N |[][S]U |[][S]F |[][S]L |[][S]C |[][S]L |.
        |[][S]K |[][S]Y |[][S]C |[][S]A |[][S]C |[][S]G |[][S]A |[][S]S |[][S]L |[][S]H |[][S]H |[][S]R |[][S]S |[][S]E |.
        |[][S]S |[][S]L |[][S]T |[][S]L |[][S]E |[][S]P |[][S]T |[][S]S |[][S]I |[][S]L |[][S]N |[][S]P |[][S]L |[][S]L |.
        |[][S]C |[][S]L |[][S]T |[][S]T |[][S]R |[][S]S |[][S]C |[][S]H |[][S]E |[][S]M |[][S]E |[][S]I |[][S]F |[][S]I |.
  \end{Puzzle}
\end{minipage}

\begin{abstract}
    A word search generator was constructed using Haskell for CS 457,
    Functional Programming, by Thomas Schreiber as his term project. Word 
    puzzles are interesting problems to solve using a functional language, and 
    in many ways functional programming is clearer in its solutions for these 
    types of programming challenges. This word search generator, Haskword, 
    takes a list of words and generates a dynamic word search puzzle. A lot of 
    research also went into IO, randomness, and randomized algorithms.

  \vspace{36pt}

  \begin{Puzzle}{14}{14}%
        |[][S]R |[][S]A |[][S]S |[][S]H |[][S]E |[][S]L |[][S]P |[][S]S |[][S]L |[][S]M |[][S]N |[][S]C |[][S]D |[][S]L |.
        |[][S]N |[][S]C |[][S]R |[][SO]H |[][SO]A |[][SO]S |[][SO]K |[][SO]E |[][SO]L |[][SO]L |[][S]R |[][S]T |[][S]S |[][S]A |.
        |[][S]H |[][S]H |[][S]L |[][S]U |[][S]N |[][S]N |[][S]R |[][S]C |[][S]A |[][S]L |[][S]U |[][S]N |[][SO]L |[][S]A |.
        |[][S]U |[][S]E |[][S]M |[][S]U |[][S]R |[][S]L |[][SO]R |[][S]P |[][S]F |[][S]R |[][S]E |[][SO]I |[][S]T |[][S]U |.
        |[][S]C |[][S]D |[][S]M |[][S]N |[][S]A |[][SO]O |[][S]L |[][S]S |[][S]I |[][S]E |[][SO]S |[][S]N |[][S]A |[][S]A |.
        |[][S]E |[][S]C |[][S]Y |[][S]N |[][SO]T |[][SO]C |[][S]U |[][S]R |[][S]I |[][SO]P |[][S]A |[][S]M |[][S]U |[][S]T |.
        |[][S]L |[][S]R |[][S]R |[][SO]C |[][S]C |[][S]I |[][SO]U |[][S]P |[][SO]G |[][SO]N |[][SO]A |[][SO]L |[][SO]R |[][SO]E |.
        |[][SO]D |[][SO]A |[][SO]N |[][SO]O |[][SO]M |[][S]T |[][S]A |[][SO]R |[][S]F |[][S]S |[][S]L |[][S]R |[][S]A |[][S]L |.
        |[][S]N |[][SO]U |[][S]E |[][S]R |[][S]E |[][S]F |[][S]N |[][S]U |[][SO]R |[][S]C |[][S]I |[][S]M |[][S]K |[][S]K |.
        |[][SO]F |[][S]T |[][S]H |[][S]A |[][S]T |[][S]S |[][S]E |[][S]I |[][S]F |[][SO]Y |[][S]R |[][S]A |[][S]S |[][S]L |.
        |[][S]L |[][SO]L |[][SO]A |[][SO]N |[][SO]O |[][SO]I |[][SO]T |[][SO]C |[][SO]N |[][SO]U |[][SO]F |[][S]L |[][S]C |[][S]L |.
        |[][S]K |[][S]Y |[][S]C |[][S]A |[][S]C |[][S]G |[][S]A |[][S]S |[][S]L |[][S]H |[][S]H |[][S]R |[][S]S |[][S]E |.
        |[][S]S |[][S]L |[][S]T |[][S]L |[][S]E |[][S]P |[][SO]T |[][SO]S |[][SO]I |[][SO]L |[][S]N |[][S]P |[][S]L |[][S]L |.
        |[][S]C |[][S]L |[][S]T |[][S]T |[][S]R |[][SO]S |[][SO]C |[][SO]H |[][SO]E |[][SO]M |[][SO]E |[][S]I |[][S]F |[][S]I |.
  \end{Puzzle}
\end{abstract}

\chapter*{Word Puzzles}     % section 1.1
    
\section*{Problem Description}       % subsection 1.1.1

    \ \ \ \ \ Word search puzzles (also known as word finds, word slueths, word seeks,
    and mystery words) are puzzles in which you are presented a grid of letters
    and instructed to find words placed inside the grid. The words could be
    placed left to right, right to left, vertically in either direction, or
    any of the four diagnals. These puzzles orginated in a Safeway grocery
    circular in the late 1960's where they were noticed as a vocabulary
    teaching aid and a new puzzle mediam to be syndicated. \cite{wiki}
    \cite{justws}

    \vspace{12pt}

    \begin{SCfigure}
        \begin{Puzzle}{7}{7}
            |[][S]O |[][S]E |[][S]L |[][S]H |[][S]L |[][S]S |[][S]E |.
            |[][S]K |[][S]S |[][S]R |[][S]H |[][S]O |[][S]S |[][S]A |.
            |[][S]O |[][S]E |[][S]O |[][S]A |[][S]E |[][S]M |[][S]B |.
            |[][S]O |[][S]S |[][S]O |[][S]E |[][S]L |[][S]B |[][S]O |.
            |[][S]B |[][S]U |[][S]B |[][S]L |[][S]A |[][S]O |[][S]L |.
            |[][S]O |[][S]O |[][S]O |[][S]H |[][S]U |[][S]M |[][S]M |.
            |[][S]R |[][S]H |[][S]E |[][S]A |[][S]S |[][S]B |[][S]E |.
        \end{Puzzle}
        \caption{A simple word puzzle, similar to those found in childeren's
                 books, that contains the hidden words: BOOK, BEES, MOLAR, and
                 HOUSE.}
    \end{SCfigure}

    A good word search puzzle has the following characteristics:

    \begin{itemize}
        \item It contains non-trivial words
        \item Words are dispursed equally around the grid
        \item Words can intersect other words
        \item Words are placed in many different directions
        \item Filler characters are similar to hidden words
    \end{itemize}

    A puzzle that has these attributes will be challenging for the solver. They
    will be required to search through the entirety of the grid, while cheking
    in every direction, for words that not only are non-trivial, but are 
    hidden among similar characters and partial substrings of the very words
    the solver is looking for. Also, as the words are solved the used letters
    cannot be ruled out as they could be used in other needed words. All of
    these characteristics make a word search much harder to solve, and we want
    to create a challenge.


\section*{Word Search vs Crossword}
    \ \ \ \ \ Originally, generating crossword puzzles were one of the goals of
    this project. Crossword puzzles are similar to word searches in a few ways
    including: across and down word placement, word intersections, and a square
    letter grid. However, crossword puzzles are quite different from word
    searches in many other ways. For example, if two words are place parallel
    and in adjacent rows then every cross section of the two words must also be
    a word. In a crossword the position of the first character of every word
    needs to be stored and later labeled. Also, the number of intersections
    needed to make an interesting crossword puzzle needs to be very high. All
    of these factors make generating an interesting crossword very complex and
    processing intensive. For these reasons I decided to focus on generating
    quality word searches for this project.

    \begin{figure}[h!]
        \begin{Puzzle}{7}{7}
            |[1][f]O |[][f]E |[][f]L |[2][f]H |[][*]L |[3][f]S |[][*]E |.
            |[][f]K |[][*]S |[][*]R |[4][f]H |[][f]O |[][f]S |[][f]A |.
            |[5][f]O |[6][f]E |[7][f]O |[][f]A |[][*]E |[][f]M |[][*]B |.
            |[][*]O |[8][f]S |[][f]O |[][*]E |[][*]L |[9][f]B |[][f]O |.
            |[10][f]B |[][f]U |[][f]B |[11][f]L |[][f]A |[][f]O |[][*]L |.
            |[][f]O |[][*]O |[][*]O |[][f]H |[][*]U |[][f]M |[][*]M |.
            |[][f]R |[][*]H |[12][f]E |[][f]A |[][f]S |[][f]B |[][f]E |.
        \end{Puzzle}
        \caption{A simple crossword puzzle.}
    \end{figure}


\section*{Deliverables}
    \ \ \ \ \ The finished product of this project will be directed at a user
    who wants to create word searches with a specific set of words. A common
    use case for this software could be generating vocabulary sets for grade 
    school classes. The user of the software would be the class teacher, and a
    number of students would be assigned to solve the puzzles.

    \vspace{12pt}

    So, at completion of this project the software will be able to take a list
    of between one and thirty words of various length, and output a quality
    word search. As this is used in a teaching environment generating unique
    random word searches for each student based off of the same word list would
    be a useful feature. It would also be helpful to output word search 
    solutions and the word list for each puzzle. The output should be good 
    looking and easy to print. The puzzle generation should run in a reasonable
    amount of time. If all of these conditions are met this project will be a
    reasonable word search generator.

\chapter*{Implementation}     % section 2.1

\section*{Data Structures}         % subsection 2.1.1
    \ \ \ \ \ We are now ready to begin to write the Haskell code for this
    project. A word search has a number of different components that we will
    need to use data structures to represent. There is the list of words that
    will be input, a grid of characters, and substrings of the grid that
    represent a way to place a word in one of the eight directions. Having a
    synonym for the points in the grid to insert words will also be helpful.

    \vspace{12pt}

    \begin{lstlisting}    
module Wordsearch where

import System.Random

type Grid    = [String]
type Point   = (Int, Int)
    \end{lstlisting}

    \vspace{12pt}

    A word search grid is represented as a list of lines, and a line is 
    nothing more than a String. The list of words makes a lot of sense as a 
    list of Strings as well. We also declare the type Point as shorthand for
    a pair of Ints. This leaves only the substrings hidden in a grid, and these
    will need to be stored as a collection of characters. Most of the time it
    will be clearer to deal with these substrings inside of the functions
    that interact with the grid itself.

    \vspace{12pt}

    Each character in the grid can be in one of two states. It can be filled or
    empty. If a space in the grid is taken by a character in one of the words
    that were input then that space is in the 'filled' state. Otherwise, the
    space is in the 'empty' state. Word searches work best with only capital
    letters. So, an octothorpe is used as to indicate that a character is in
    an 'empty' state.

    \vspace{12pt}

    One task that we will need to do is create a new grid of a given size. This
    grid will have all characters in the 'empty' state.

    \vspace{12pt}

    \begin{lstlisting}
emptyGrid     :: Int -> Int -> Grid
emptyGrid  x y = replicate y [s | s <- (replicate x '#')] 
    \end{lstlisting}

    \vspace{12pt}

    We will also need to be able to display the contents of a grid:

    \vspace{12pt}

    \begin{lstlisting}
displayGrid :: Grid -> IO ()
displayGrid  = putStr . unlines
    \end{lstlisting}

    \vspace{12pt}

    With just the few type synonyms listed above the word search components can
    now be used. In this way the built-in types in Haskell (specifically
    Strings and lists) are very well suited to the problem of generating word
    searches. As Strings are just lists of characters we will be able to take
    advantage of a lot of the built-in functions that operate on lists as we
    write the rest of the implementation of this project.

\section*{Word Insertion}
    \ \ \ \ \ Now that we can generate an empty grid we need to add words to
    it. The most basic way to add a word is forward and across. We want to add
    a word at a given point in the grid.

    \vspace{12pt}

    \begin{lstlisting}
insertAWord          :: Grid -> String -> Point -> Grid
insertAWord g s (x,y) = take y g ++ ((take x (g !! y)) 
                                 ++ s 
                                 ++ (drop (length s + x)) (g !! y)) 
                                  : (drop (y+1) g)
    \end{lstlisting}

    \vspace{12pt}

    The function insertAWord takes a Grid, String, and Point as input. Then the
    String, s, is added to the grid at point (x,y). This is accomplished by
    taking all of the lines of the input grid that will not contain the line
    that s is being inserted into. These lines are concatonated onto a line
    composed of the characters that come before s, s, and the characters that
    come after s on that line. Finall, we concatonate all of the lines after
    the line that we added s to. This creates a new Grid that now has s
    inserted at the correct spot.

    \vspace{12pt}

\begin{lstlisting}
*Wordsearch> displayGrid $ insertAWord (emptyGrid 10 8) "HELLO" (3,3)
##########
##########
##########
###HELLO##
##########
##########
##########
##########
    \end{lstlisting}

    \vspace{12pt}

    The next function that needs to be implemented will be used to insert words
    vertically down.

    \vspace{12pt}
    \pagebreak

    \begin{lstlisting}
insertDWord          :: Grid -> String -> Point -> Grid
insertDWord g s (x,y) = take y g 
                        ++ (map (\c -> (take x (g !! snd c)) 
                        ++ (fst c) : drop (x + 1) (g !! snd c)) (zip s [y..])) 
                        ++ drop (length s + y) g
    \end{lstlisting}

    \vspace{12pt}

    This function is similar to insertAWord. However, insertDWord operates on a
    list of lines. one character is replaced in each line with the correct 
    character from the string s. We can now add a word vertically to a grid, 
    even a grid that already has a word in it.

    \vspace{12pt}

    \begin{lstlisting}
*Wordsearch> let hello = insertAWord (emptyGrid 10 8) "HELLO" (3,3)
*Wordsearch> displayGrid $ insertDWord hello "WORLD" (7,2)
##########
##########
#######W##
###HELLO##
#######R##
#######L##
#######D##
##########
    \end{lstlisting}

    \vspace{12pt}

    Of course we will want to combine the two functions, insertAWord and 
    insertDWord, into a function that will insert words sideways.

    \vspace{12pt}

    \begin{lstlisting}
insertSWWord1          :: Grid -> String -> Point -> Grid
insertSWWord1 g s (x,y) = take y g 
                        ++ (map (\c -> (take (x + snd c - y) (g !! snd c)) 
                        ++ (fst c) : drop (x + (snd c) + 1 - y) (g !! snd c)) 
                           (zip s [y..])) 
                        ++ drop (length s + y) g
    \end{lstlisting}

    \vspace{12pt}

    While insertSWWord1 can seem a bit more complicated than insertAWord or 
    insertDWord, but it works in a very similar way. We are still taking the
    lines that appear above where the word will be inserted. Then, operating
    on the lines from the point of insertion to the length of the word s. Now,
    there is a difference though, because we also must increment the x
    position after each character is added so that the string, s, is inserted
    diagnally. A lambda function is used to insert the correct letter in the 
    correct line at the correct position. The insertion point of each character 
    is incremented by one line and one position to the right inside of each 
    line.

    \vspace{12pt}
    \pagebreak

    \begin{lstlisting}
*Wordsearch> let helloworld = insertDWord hello "WORLD" (7,2) 
*Wordsearch> rawDisplay $ insertSWWord1 helloworld "HASKELL" (0,1)
##########
H#########
#A#####W##
##SHELLO##
###K###R##
####E##L##
#####L#D##
######L###
    \end{lstlisting}

    \vspace{12pt}

    We can derive the remaining 5 ways to insert words (horizontally-backwards,
    vertically-up, diagnally-up-left-to-right, diagnally-up-right-to-left, and
    diagnally-down-right-to-left) by taking advantage of the Haskell prelude
    function reverse. With various combinations of reversing the input string
    and reversing the lines of the grid all of the remaining ways to add words
    to a grid are trivial to write.

    \vspace{12pt}

    If we carefully plan out a word search the insertion functions can now be
    used to place them into a grid. For the purposes of this program though we
    need to be able to check if the word fits within the bounds of the grid,
    and we also need to be able to prevent clobbering letters on the board
    that are part of previously added words.

    \vspace{12pt}

    \begin{lstlisting}
fitsA          :: Grid -> String -> Point -> Bool
fitsA g s (x,y) = if (x + length s) <= (length . head) g then
                  foldr (&&) True (map (\c -> fst c == snd c || snd c == '#') 
                        (zip s (take (length s) (drop x (g !! y)))))
                  else False

fitsD          :: Grid -> String -> Point -> Bool
fitsD g s (x,y) = if (y + length s) <= length g then
                  foldr (&&) True (map (\c -> fst c == snd c || snd c == '#') 
                        (zip s (map (!!x) (take (length s) (drop y g)))))
                  else False

fitsSW1          :: Grid -> String -> Point -> Bool
fitsSW1 g s (x,y) = if (x + length s) <= (length . head) g 
                    && (y + length s) <= length g then 
                    foldr (&&) True (map (\c -> fst c == snd c || snd c == '#') 
                        (zip s (map 
                                (\c -> (g !! snd c) !! (x + (snd c) - y)) 
                                (zip s [y..]))))
                    else False
    \end{lstlisting}

    \vspace{12pt}

    Again we can derive the other five functions by using reverse. All of these
    functions work in a very similar way to the insert functions with a few
    changes. If the words go beyond the bounds of the board it returns False.
    Also, if any of the characters in the grid are not '\#', or the character
    we are trying to place, False is returned to prevent clobbering the other
    words. Otherwise, True is returned.

\section*{Randomness}
    \ \ \ \ \ One of the requirements that was set for this project is that it
    should create random word searches based off of the input. One solution to
    creating an algorithm to solve this poblem is to randomly place the words
    one by one. If the word fits in a randomly chosen place then we add it to
    the grid. Otherwise, we try to add it at another random location.

    \vspace{12pt}

    There is one big downside to this method however. It is possible that the
    algorithm will continue to randomly choose locations where the word that
    is being inserted does not fit. If we choose an appropriate grid size the
    probability of the algorithm running forever (or even long amounts of time)
    becomes very low. This algorithm will be very similar to so called 'balls
    and bins' problems. A famous example of this type of problem is the called
    the birthday problem. Given a room of $n$ people what is the probability 
    that at least two of the people share a birthday? If we think of the days
    of the year as bins, and birthdays as balls, what we want to know is the
    probability that there are two or more balls in any one bin.

    \vspace{12pt}

    We know that there are 366 possible birthdays (don't forget leap years). 
    So, we can calculate the number of ways $n$ dates can be chosen from 366.
    We can assign these birthdays to the $n$ people in $n!$ ways. So, there
    are $ {{366} \choose {n}} n!$ ways for everyone to have different birthdays.
    We can also calculate the total number of possible birthday configurations
    as $366^{n}$. So, the probability, p, that at least two people share a
    birthday is:

    \vspace{12pt}
    \begin{center}
        $ p = {{{366 \choose n} n!} \over {366^{n}}} $
    \end{center}

    \vspace{12pt}

    If we think about our word search algorithm as a 'balls and bins' problem,
    then we can see that we can lower the probability of character collisions
    by increasing the size of the board. We must only think of the positions in
    the grid as the bins and the letters in the words as balls. In a future
    revision of this project it would be good to find a good probability that 
    meets the run time constraints of the users, and warn users if the grid
    size they request is too small. \cite{prob}

    \vspace{12pt}

    Haskell has a module for doing tasks that require randomness. The module is
    especially interesting, because there are IO and non-IO functions for many
    tasks. As Haskell is a pure functional language the non-IO functions must
    be seeded. While either method (seeded or un-seeded) will work fine for
    this program I decided to go with the un-seeded IO functions.

    \vspace{12pt}

    The most straight forward function from the System.IO module is randomRIO.
    randomRIO is defined as:

    \vspace{12pt}

    \begin{lstlisting}
randomRIO :: (a, a) -> IO a
    \end{lstlisting}

    \vspace{12pt}

    This function takes a pair of a similar type, and then randomly returns
    a value between the first element of the pair and the second element of the
    pair. Also, note that randomRIO returns an IO a. So, we will need to keep
    this in mind as we create functions that utilize the randomRIO function.

    \vspace{12pt}

    One function that we will need to write is a function that can randomly
    pick an element from a given list. This can be done by using randomRIO to
    get a random index of one of the elements. Then, we can take just that one
    element from the list.

    \vspace{12pt}

    \begin{lstlisting}
pick   :: [a] -> IO a
pick xs = randomRIO (0, (length xs - 1)) >>= return . (xs !!)
    \end{lstlisting}

    \vspace{12pt}

    Now, we can pick elements from a list. One list we will want to pick
    elements from is a list of all of the ways to insert a word. These
    elements should be paired with their respective fit functions.

    \vspace{12pt}

    \begin{lstlisting}
waysToInsert = [(fitsA, insertAWord), (fitsAR, insertAWordR),
                (fitsD, insertDWord), (fitsDR, insertDWordR),
                (fitsSW1, insertSWWord1), (fitsSW1R, insertSWWord1R),
                (fitsSW2, insertSWWord2), (fitsSW2R, insertSWWord2R)]

randomWay  = pick waysToInsert
    \end{lstlisting}

    \vspace{12pt}

    Now, everything is ready to write the function to insert a word into a 
    random place (in a random direction) in a given grid. To do this we use
    randomWay. Then, we use randomRIO to select the x and y values for the 
    point of insertion. Finally, we check to see if the word will fit, and if
    it does then we insert it. If the word does not fit we recursively call
    insert word and try a new random way to add the word.

    \vspace{12pt}

    \begin{lstlisting}
insertWord    :: Grid -> String -> IO Grid
insertWord grid w = do method <- randomWay
                       x <- randomRIO (0, length (head grid) - 1)
                       y <- randomRIO (0, length grid - 1)
                       if (fst method) grid w (x,y)
                         then return $ ((snd method) grid w (x,y))
                         else insertWord grid w

insertWords         :: Grid -> [String] -> IO Grid
insertWords g []     = return g
insertWords g (x:xs) = do newGrid <- insertWord g x
                          insertWords newGrid xs
    \end{lstlisting}

    \vspace{12pt}

    We can now create a randomized grid with the words we have selected.

    \vspace{12pt}

    \begin{lstlisting}
*Wordsearch> let testWords = ["ONE", "TWO", "THREE"]
*Wordsearch> insertWords (emptyGrid 5 5) testWords >>= displayGrid
####T
###HE
##RN#
#EOWT
E####
    \end{lstlisting}

    \vspace{12pt}

    This is starting to look like a word search. In fact this is a great way
    to produce solutions for word searches. These Grids with only letters from
    the inputed words can be zipped with the completed Grids to make a pair
    for printing puzzles and keys.

    \vspace{12pt}

    There is still one more task that must be completed before the puzzles can
    be generated. All of the '\#' must be replaced by random letters. We don't
    want completely random characters, because one of the features of good 
    word searches is having partial substrings of the words mixed amongst the
    characters. Ideally, the non-word characters will appear at the same
    frequency they do in the input words. We can use pick to select a random
    character uniformly from the set of characters that make up the input
    words.

    \vspace{12pt}

    \begin{lstlisting}
replaceHash     :: [String] -> Char -> IO Char
replaceHash ws c = if c == '#' then do r <- pick (concat ws)
                                       return r
                               else return c

fillLine     :: [String] -> String -> IO String
fillLine ws l = mapM (replaceHash ws) l

fillGrid     :: [String] -> Grid -> IO Grid
fillGrid ws g = mapM (fillLine ws) g
    \end{lstlisting}

    \vspace{12pt}

    This collection of functions will use the mapM function to operate on each
    character. The replaceHash function checks if each individual position in
    the grid is empty (denoted by the '\#' symbol). It then uses pick to 
    uniformly select one of the characters from the input words and relplace
    the hash. We now have a proper word search.

    \vspace{12pt}

    We will need to alter displayGrid slightly to fill and show our generated
    word searches.

    \vspace {12pt}

    \begin{lstlisting}
displayGrid  :: IO Grid -> [String] -> IO ()
displayGrid g ws = do grid <- g
                      filledGrid <- fillGrid ws grid
                      putStr (unlines filledGrid)
    \end{lstlisting}

    \pagebreak

    \begin{lstlisting}
*Wordsearch> let ws = ["ONE", "TWO", "THREE"]
*Wordsearch> displayGrid (insertWords (emptyGrid 5 5) ws) ws
TOHET
OHNHO
NRRNT
EERTN
EWTWO
    \end{lstlisting}


\section*{IO}

\chapter*{Observations}
\section*{Functional Programming}
\section*{Evaluation of This Implementation}
\section*{Word Puzzles Revisited}

\begin{thebibliography}{9}
  % type bibliography here
    \bibitem{learnyou}
        Miran Lipovaca.
        \newblock {\em Learn You a Haskell for Great Good!}
        \newblock No Starch Press, San Francisco, CA, 2011.

    \bibitem{prob}
        Michael Mitzenmacher and Eli Upfal
        \newblock {\em Probability and Computing} \\
        \newblock Cambridge University Press, New York, NY, 2005.

    \bibitem{rand}
        \newblock {\em System.Random} \\
        \newblock \url{http://www.haskell.org/ghc/docs/6.12.1/html/libraries/random-1.0.0.2/System-Random.html}, March 10, 2012.

    \bibitem{justws}
        \newblock {\em The Histroy of Word Search}
        \newblock JustWordSearch.com, March 10, 2012.

    \bibitem{wiki}
        \newblock {\em Word search}
        \newblock \url{http://en.wikipedia.org/wiki/Word_search}, March 10, 2012.

\end{thebibliography}


\end{document}
